\documentclass[xcolor=table,aspectratio=169,dvipsnames,english]{beamer}
\usepackage{bm}
\usepackage[utf8]{inputenc}
\usepackage{color}

\usepackage[british]{babel} % decent hyphenation, avoiding e.g. anal-ysis
\usepackage[iso]{isodate}
\usepackage{sansmath}
\usepackage{booktabs}
\usepackage{graphicx}
\usepackage{graphviz}
\usepackage{makecell}
\usepackage{minted}
\usepackage{multicol}
\usepackage{siunitx}
\usepackage{subcaption}
\usepackage[section]{placeins}

% Needs to be loaded after hyperref
\usepackage{cleveref}

% PythonTeX
\usepackage[autoprint=false, gobble=auto, keeptemps=all, pyfuture=all]{pythontex} % create figures on-line directly from python!
\usepackage{pgf}
%\input{/usr/share/repsep/functions.py}
\input{functions.py}
\begin{pythontexcustomcode}[begin]{py}
pytex.add_dependencies(
	'lib/utils.py',
	'lib/categorical.py',
	'data/JogB.tsv'
	)
\end{pythontexcustomcode}
% Single-session PythonTeX codeblocks
\newcounter{pysessioncounter}
\newcommand{\sessionpy}{%
          \edef\sessionpysession{session\arabic{pysessioncounter}}%
            \stepcounter{pysessioncounter}%
              \expandafter\py\expandafter[\sessionpysession]}

% SIunitx customizations detect-all will use the current font for typesetting
\sisetup{per-mode=symbol, detect-all, range-units = single}
\newcommand\SIci[5]{\SI{#1}{#2}, {#3}CI: \SIrange{#4}{#5}{#2}}

% Fix for matplotlib PGF wonkiness which isn't interpreted correctly by pdflatex
\DeclareUnicodeCharacter{2212}{-}


%\hypersetup{
%    colorlinks=true,
%    urlcolor=cyan
%}

%BIBLIOGRAPHY
% see https://mirror.foobar.to/CTAN/macros/latex/contrib/biblatex/doc/biblatex.pdf for available styles
\usepackage[backend=bibtex,style=numeric,natbib=true]{biblatex}
\addbibresource{bib.bib}
\addbibresource{old_bib.bib}

% Article-specific configuration
\begin{pythontexcustomcode}[begin]{py}
DOC_STYLE="slides/main.conf"
pytex.add_dependencies(
	DOC_STYLE,
	'slides/1col.conf',
	)
\end{pythontexcustomcode}

% Custom beamer styling and colors
\setbeamersize{text margin left=0.8em,text margin right=0.8em}
\setbeamertemplate{bibliography item}{\insertbiblabel}

\usecolortheme[RGB={199,199,199}]{structure}
\usetheme{Dresden}

\captionsetup[figure]{labelformat=empty}

\definecolor{dy}{RGB}{202,202,0}
\definecolor{rsblue}{HTML}{00a3cc}
\definecolor{mg}{gray}{0.30}
\definecolor{lg}{gray}{0.60}
\definecolor{vlg}{gray}{0.78}
\definecolor{tlg}{gray}{0.88}

\setbeamercolor{caption name}{fg=lg}
\setbeamercolor{caption}{fg=lg}
\setbeamercolor{author}{fg=lg}
\setbeamercolor{institute}{fg=lg}
\setbeamercolor{date}{fg=lg}
\setbeamercolor{title}{fg=mg}
\setbeamertemplate{caption}{\centering\insertcaption\par}
\setbeamertemplate{navigation symbols}{}

% Navigation symbols are too far down by default
% To further adjust edit the pt numbers before and after `\insertnavigation`
\makeatletter
\defbeamertemplate*{headline}{my miniframes theme}
{%
        \begin{beamercolorbox}[colsep=1.5pt]{upper separation line head}
        \end{beamercolorbox}
        \begin{beamercolorbox}{section in head/foot}
                \vskip1pt\insertnavigation{\paperwidth}\vskip3pt
        \end{beamercolorbox}%
        \ifbeamer@theme@subsection%
                \begin{beamercolorbox}[colsep=1.5pt]{middle separation line head}
                \end{beamercolorbox}
                \begin{beamercolorbox}[ht=2.5ex,dp=1.125ex,%
                        leftskip=.3cm,rightskip=.3cm plus1fil]{subsection in head/foot}
                        \usebeamerfont{subsection in head/foot}\insertsubsectionhead
                \end{beamercolorbox}%
        \fi%
        \begin{beamercolorbox}[colsep=1.5pt]{lower separation line head}
        \end{beamercolorbox}
}
\makeatother


\title{Multimodal Generative Learning on the MIMIC-CXR Database}
\subtitle{A presentation of my semester project}
\author{Hendrik Klug}
\institute{Institute for Electrical Engineering, ETH}
\begin{document}
	\begin{frame}
		\titlepage
	\end{frame}
	\section{Introduction}

	    \begin{frame}
	        In this work, we applied a method for self-supervised, multimodal and generative training from \cite{thomas_multimodal} on the MIMIC-CXR Database \cite{johnson2019mimic}.
	    \end{frame}

		\subsection{Multimodal, Unsupervised, Generative models}
			\begin{frame}{The General Idea}
				\py{pytex_printonly(script='scripts/generalidea_graph.py', data = '')}
			\end{frame}

        \begin{frame}{Multimodal, Unsupervised Generative Learning On Medical Data}
            \begin{itemize}
                \item No need for labeled data
                \item Can extract features from multiple modalities
                \item Can generate \textit{coherent} samples from one input modality
            \end{itemize}
        \end{frame}

    \section{Background}
        \subsection{The MoPoE-VAE}
            \begin{frame}{The Mixture-of-Products-of-Experts-VAE}
            Combination of:
                \begin{itemize}
                    \item The Product-of-Experts (PoE) from \cite{wu2018multimodal}
                    \item The Mixture-of-Experts (MoE) from \cite{shi2019variational}
                \end{itemize}
                \vspace{\baselineskip}
                Both differ in their choice of the joint posterior approximation functions.
            \end{frame}

            \begin{frame}{The PoE-VAE}

                \begin{itemize}
                    \item Uses a geometric mean: the joint posterior is a product of individual posteriors
                    %p(z|x_1,...,x_N)) \propto p(z) \prod _{i=1} ^N \tilde{q}(z|x_i)
                \begin{equation}
                    q_{\Phi}(z|x_{1:M})=\prod _m q_{\Phi_m}(z|x_m)
                \end{equation}
                    \item Results in a good approximation of the joint distribution but struggles in optimizing the individual experts.
                \end{itemize}

            \end{frame}

            \begin{frame}{The MoE-VAE}
            \begin{itemize}
                \item Uses an arithmetic mean
                \begin{equation}
                    q_{\Phi}(z|x_{1:M})=\sum _m \alpha_m\cdot q_{\Phi_m}(z|x_m)
                \end{equation}
                \item Optimizes individual experts well but is not able to learn a distribution that is sharper than any of its experts.
            \end{itemize}

            \end{frame}

            \begin{frame}{The Mixture-of-Products-of-Experts-VAE}
                The generalized multimodal ELBO utilizes the PoE to get the posterior approximation of a subset $\xsubset \in \mathcal{P}(\mathbb{X})$:

				\begin{equation}
					\tilde{q}_{\phi}(\textbf{z}|\xsubset)=PoE(\{q_{\phi_j}(\textbf{z}|\textbf{x}_j) \forall \textbf{x}_j \in \xsubset\}) \propto \prod _{\textbf{x}_j \in \xsubset}q_{\phi_j}(\textbf{z}|\textbf{x}_j)
				\end{equation}
				And the MoE to get the joint posterior:
				\begin{equation}
					q_{\phi}(\textbf{z}|\mathbb{X}) = \frac{1}{2^3} \sum _{\textbf{x}_k \in \mathbb{X}} \tilde{q}_{\phi} (\textbf{z}|\mathbb{X}_k)
				\end{equation}
            \end{frame}

            \begin{frame}{Frame Title}

			\py{pytex_printonly(script='scripts/mopoe_graph.py', data = '')}

            \end{frame}

    \section{Methods}
        \subsubsection{The MIMIC-CXR Database}
            \begin{frame}{The dataset}
            The MIMIC-CXR Database \cite{johnson2019mimic} is a large publicly available dataset of chest radiographs with free-text radiology reports containing 377,110 images corresponding to 227,835 radiographic studies performed at the Beth Israel Deaconess Medical Center in Boston, MA.
                
            \end{frame}
            
            \begin{frame}
                \begin{enumerate}
                    \item Implemented word encoding (show dictionary example)
                    \item tested image size
                    \item tested beta
                    \item tested class dim
                \end{enumerate}
            \end{frame}

            \begin{frame}
                \py{pytex_fig('scripts/rand_sample_from_dataset.py', conf='slides/dataset.conf', label='dataset_samples', caption='Samples from the dataset.')}
            \end{frame}
            
            \begin{frame}{The labels}
                Since the multiple labels from the dataset are highly imbalanced, we created a label "Finding", indicating if the sample is labeled with any pathology.\\
                There are \py{boilerplate.print_finding_counts()} samples that are annotated with the "Finding" label in the training set and \py{boilerplate.print_nofinding_counts()} samples that are not.
            \end{frame}
            
            \begin{frame}{The word encoding}
                Every word that occurs at least \py{boilerplate.print_flag_attribute('word_min_occ')} times in all the text reports is mapped to an index.\\
                Using this mapping each sentence is encoded into a sequence of indices.\\
                
                \pause
                "Heart size is normal." $\rightarrow [0,1,2,3] \rightarrow$ Model $\rightarrow
                \begin{bmatrix}
                1 & 0 & 0 & 0\\
                0 & 1 & 0 & 0\\
                0 & 0 & 1 & 0\\
                0 & 0 & 0 & 1\\
                \end{bmatrix}$\\
                $\rightarrow [0,1,2,3] \rightarrow$ "Heart size is normal." 
            \end{frame}
            
        \subsection{The model architecture}
        \begin{frame}{The model}
            The MoPoE-VAE has an encoder and decoder for each modality, in this work both are the same for the image modalities (frontal and lateral scans).
        \end{frame}
        
        \begin{frame}{Resnet}
        % todo
        Show resnet architecture
            
        \end{frame}
        
        \subsection{Evaluation methods}
        \begin{frame}{Evaluation of the latent representation}
        \begin{figure}[!h]
        \centering
        \begin{tikzpicture}
        
              \begin{scope}[blend group = soft light]
                \fill[blue!40!white]   (0:0) ellipse (6cm and 3cm);
                \fill[green!40!white] (0:-2) ellipse (1.5cm and 2cm);
                \fill[red!40!white]  (0:2) ellipse (1.5cm and 2cm);
                \draw [-,ultra thick] (0,-1.5) -- (0,1.5);
            \end{scope}
                \node at (0:-2)   {"No Finding"};
              \node at (0:2)   {"Finding"};
              \node at (90:2.3) [font=\Large] {Latent Representation};
            \end{tikzpicture}
        \end{figure}    
        \end{frame}
        
        \begin{frame}{Evaluation of the generation coherence}
        To evaluate the coherence of the generated samples, they are classified using trained ResNet classifiers.
        If all modalities of a sample, generated and given as conditioner, are classified as having the same label, they are considered coherent.
        \end{frame}
        
    \section{Results}
        \subsection{Generation Evaluation}
        
    
        
\printbibliography
\end{document}
