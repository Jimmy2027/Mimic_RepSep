\documentclass{midl} % Include author names
%\documentclass[anon]{midl} % Anonymized submission

% The following packages will be automatically loaded:
% jmlr, amsmath, amssymb, natbib, graphicx, url, algorithm2e
% ifoddpage, relsize and probably more
% make sure they are installed with your latex distribution

\usepackage{mwe} % to get dummy images

% Header for extended abstracts
\jmlrproceedings{MIDL}{Medical Imaging with Deep Learning}
\jmlrpages{}
\jmlryear{2021}

% to be uncommented for submissions under review
\jmlrworkshop{Short Paper -- MIDL 2021 submission}
\jmlrvolume{-- Under Review}
\editors{Under Review for MIDL 2021}

\title[Multimodal Generative Learning on the MIMIC-CXR Database]{Multimodal Generative Learning on the MIMIC-CXR Database}


% More complicate cases, e.g. with dual affiliations and joint authorship
\midlauthor{\Name{Hendrik Klug\nametag{$^{1,2}$}} \Email{klugh@ethz.ch}\\
\addr $^{1}$ Institute for Electrical Engineering, ETH Zürich \\
\Name{Thomas M. Sutter\midlotherjointauthor\nametag{$^{2}$}} \Email{xyz@sample.edu}\\
\addr $^{2}$ Department of Computer Science, ETH Zürich \AND
\Name{Julia E. Vogt\nametag{$^{2}$}} \Email{alphabeta@example.edu}\\
}

\begin{document}

\maketitle
\begin{abstract}
    Machine Learning (ML) has become more and more popular in the medical domain over the past years.
While supervised machine learning has received the most attention, unsupervised learning has great potential since there is more data available for the training of the models.
Especially when leveraging the multi-modal nature of most clinical data, self-supervised learning can provide results comparable to supervised methods.
    Multi-modal unsupervised methods have been tested extensively on toy-datasets like MNIST \cite{lecun-mnisthandwrittendigit-2010, thomas_gener-ELBO, wu2018multimodal} and CelebA \cite{liu2015faceattributes, thomas_gener-ELBO, wu2018multimodal} but to the best of our knowledge have never been applied to medical data, for direct applications such as disease classification and image generation.
In this article, we apply a method for self-supervised training proposed in \cite{thomas_gener-ELBO} to the MIMIC-CXR Database \cite{johnson2019mimic} and evaluate it in an exhaustive list of experiments.
\end{abstract}




\begin{keywords}
Multimodal Learning, Generative Learning, 
\end{keywords}

\section{Introduction}

This is where the content of your paper goes.  Some random notes:
\begin{itemize}
\item You should use \LaTeX \cite{Lamport:Book:1989}.
\item JMLR/PMLR uses natbib for references. For simplicity, here, \verb|\cite|  defaults to
  parenthetical citations, i.e. \verb|\citep|. You can of course also
  use \verb|\citet| for textual citations.
\item You should follow the guidelines provided by the conference.
\item Read through the JMLR template documentation for specific \LaTeX
  usage questions.
\item Note that the JMLR template provides many handy functionalities
such as \verb|\figureref| to refer to a figure,
e.g. \figureref{fig:example},  \verb|\tableref| to refer to a table,
e.g. \tableref{tab:example} and \verb|\equationref| to refer to an equation,
e.g. \equationref{eq:example}.
\end{itemize}

    Training a deep-learning model in a supervised manner consists of updating its parameters based on observations of a large amount of labeled data.
	Manually creating annotations of sufficient training examples for the training of the model is often infeasible, since it requires manual expert input.
	In the medical domain especially, human labeled data is expensive to acquire and thus very scarce.
	A generative model, that can learn embeddings of the data without the need for labels, enjoys a much bigger variety of possible training data.
	The Variational Autoencoder (VAE) \cite{doersch2016tutorial} in particular is a popular generative model, which consists of an encoder that maps the input to a learned latent distribution from which the decoder part samples to reconstruct the input.

	Clinical data comes in many modalities, such as images, text reports and electronic health records.
	A generative model that can leverage all of the modalities could be used for tasks such as generating a text report when given images of a patient, generating an image of another angle from an input image or data classification.
	Generating a text report could relieve doctors from a tedious and time consuming task and accurate classification of disease state could accompany their judgement.
	The generation of images from multiple angles could, in theory, avoid having to take multiple scans, which is costly, time-consuming and harmful in the case of X-ray scans.

	In this work, we apply the generalized multimodal ELBO, proposed in \cite{thomas_gener-ELBO}, to train a self-supervised, generative model on clinical data from the MIMIC-CXR Database \cite{johnson2019mimic} containing chest-X rays and related text reports. In order to provide a first insight into the difficulties and oppurtunities that come with medical data.

\begin{table}[htbp]
 % The first argument is the label.
 % The caption goes in the second argument, and the table contents
 % go in the third argument.
\floatconts
  {tab:example}%
  {\caption{An Example Table}}%
  {\begin{tabular}{ll}
  \bfseries Dataset & \bfseries Result\\
  Data1 & 0.12345\\
  Data2 & 0.67890\\
  Data3 & 0.54321\\
  Data4 & 0.09876
  \end{tabular}}
\end{table}

\begin{figure}[htbp]
 % Caption and label go in the first argument and the figure contents
 % go in the second argument
\floatconts
  {fig:example}
  {\caption{Example Image}}
  {\includegraphics[width=0.5\linewidth]{example-image}}
\end{figure}

\begin{algorithm2e}
\caption{Computing Net Activation}
\label{alg:net}
 % older versions of algorithm2e have \dontprintsemicolon instead
 % of the following:
 %\DontPrintSemicolon
 % older versions of algorithm2e have \linesnumbered instead of the
 % following:
 %\LinesNumbered
\KwIn{$x_1, \ldots, x_n, w_1, \ldots, w_n$}
\KwOut{$y$, the net activation}
$y\leftarrow 0$\;
\For{$i\leftarrow 1$ \KwTo $n$}{
  $y \leftarrow y + w_i*x_i$\;
}
\end{algorithm2e}

% Acknowledgments---Will not appear in anonymized version
\midlacknowledgments{We thank a bunch of people.}


\bibliography{bib}


\appendix

\section{Proof of Theorem 1}

This is a boring technical proof of
\begin{equation}\label{eq:example}
\cos^2\theta + \sin^2\theta \equiv 1.
\end{equation}

\section{Proof of Theorem 2}

This is a complete version of a proof sketched in the main text.

\end{document}